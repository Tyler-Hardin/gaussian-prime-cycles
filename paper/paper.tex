\documentclass{article}
\usepackage{amsmath}
\usepackage{amssymb}
\usepackage{endnotes}
\usepackage{float}
\usepackage[margin=.75in]{geometry}
\usepackage{graphicx}
\usepackage{setspace}
\usepackage{subfigure}
\usepackage{url}

\doublespacing
\title{Numerical Research on Gaussian Prime Cycles}
\author{Karen Dana \and Tyler Hardin}
\begin{document}
	\maketitle
	
A Gaussian integer is a complex number restricted to integer-valued real and complex components. For example, $5+6i$ is a Gaussian integer, while $5.5+\pi i$ is not. A Gaussian prime, for the purposes of this paper, is a Gaussian integer $z$ such that one of the following is true \cite{elem_num_thry}:
% https://tex.stackexchange.com/questions/114471/how-to-simply-center-equations-using-eqnarray
\begin{gather*}
  \mbox{im}(z) \mbox{ is prime and } \mbox{real}(z) = 0 \\
  \mbox{real}(z) \mbox{ is prime and } \mbox{im}(z) = 0 \\
  \sqrt{\mbox{real}(z)^2 + \mbox{im}(z)^2} \mbox{ is prime and } \mbox{real}(z) \neq 0 \mbox{ and } \mbox{im}(z) \neq 0
\end{gather*}
  
A Gaussian prime cycle, is a path formed by starting at an initial point (e.g. $(5,7)$), we will call $x_0$, with an initial direction (eg $(1,0)$), we will call d, and stepping right (that is $x_{n+1}=x_n + d$) until $x_n$ is a Gaussian prime. If $x_n$ is a Gaussian prime, we rotate $d$ $90^{\circ}$ counterclockwise, and continue in that direction until another Gaussian prime is encountered; then we rotate again. This process continues until we return to our initial point \textit{and} have our initial direction. The cycles are not proven to exist for all initial points, and as a new problem, it may be a while until the result is known. \cite{mathoverflow,wolfram}

For our purposes, an equivalence relation is a relation that partitions a set. 
Equivalence classes are sets of elements that are all equal under a given equivalence relation. It is a known property of equivalence relations that every element belongs to exactly one class.
An example would be the natural numbers with the equivalence relation being equivalence modulus 2.
If $n$, a natural number, is even, it belongs to the class of even numbers which are all equal mod 2.
And vice versa if it is odd. Every natural number is either even or odd, so every natural number is in at least one class. No natural number is both even and odd, so no natural number can be in both classes. Thus every element of $\mathbb{N}$ is in exactly one class.
We use this concept to define an equivalence relation on the Gaussian prime cycles.
$G$, a Gaussian prime cycle, is considered equivalent to $H$, another Gaussian prime cycle, if $G$ can be rotated or translated such that the are the same shape.
More verbosely, they are equal if there exists a bijective translation-rotation transformation $T_M:G \to H$.

We plotted the number of classes with an initial point in a $l \times l$ grid as a function of $l$, with the grid being from $(0,0)$ to $(l-1,l-1)$ inclusively. Stated more simply: the number of unique shapes starting in the square on the $y$ axis and the size of the square on the $x$ axis. This allows us to see what the likely trend is as to the rate of discovering new shapes as one searches farther from $(0,0)$. The cycle may leave the square, only the initial point is considered.

\begin{figure}[H]
\center
\includegraphics[scale=.7]{clgl30.png}
\caption{Number of classes as a function of grid size ($l$). Interestingly, the relation appears to be approximately linear.}
\end{figure}

We also plotted the length of a cycle as a (multi-valued) function of the Euclidean distance from $(0,0)$. A trend in this would indicate that the farther one is from $0$, the longer the cycle one is likely to find. This was accomplished by plotting all cycles in a given grid for multiple grid sizes. Interestingly, the grouping and relative shape of the scatter plot is similar for many sizes of grid, even though the absolute difference is large. Scale invariance of the clustering of the graph would be a really interesting result.

\begin{figure}[H]
\center
\subfigure[All paths in grid of size 10.]{\includegraphics[scale=0.4]{icl10.png}}
\subfigure[All paths in grid of size 20.]{\includegraphics[scale=0.4]{icl20.png}}\\
\subfigure[All paths in grid of size 30.]{\includegraphics[scale=0.4]{icl30.png}}
\subfigure[All paths in grid of size 70.]{\includegraphics[scale=0.4]{icl70.png}}

\caption{Length of cycle vs. Euclidean distance from $(0,0)$. Notice how, in all, there are essentially three clusters.}
\end{figure}

Finally, we graphed a 2D scatter plot colored with each point $(x_i, y_j)$ colored by the class to which the cycle with initial point $(x_i, y_j)$ belongs. Unfortunately, there are no obvious patterns to the graph, so there isn't much to discuss in terms of the possible implications of this visualization. Colors are assigned from the RGB spectrum in order of cycle length. That is, the closer to red, the longer the cycle, the closer to violet, the shorter the cycle.

\begin{figure}[H]
\center
\includegraphics[scale=.7]{class_scatter.png}
\caption{Points colored by the class to which they (i.e. their path) belong.}
\end{figure}

Unfortunately, due to the deep nature of the Gaussian prime cycle problem with its relation to primes and recursive processes, there isn't much we can recommend in terms of the further, more theoretical research based on our results. It could be interesting to try to prove the correlations we describe, but likely well out of reach. In terms of further numerical research, the (possible) correlation visualized the first graph would benefit more from parallelization and more computational power than the second would, due to the algorithmic complexity of the equivalence class enumeration code.

\bibliographystyle{plain}
\bibliography{paper}

\end{document}
