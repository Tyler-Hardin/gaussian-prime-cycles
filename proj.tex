\documentclass[12pt]{article}

\author{Karen Dana \and Tyler Hardin}
\title{Project Proposal: Gaussian Prime Cycles}
\begin{document}
\maketitle

Firstly, we need to define that Gaussian integers are complex numbers of the
form $a+bi$, where $a$ and $b$ are integers. Some Gaussian integers have no
nontrivial factorization. These can be considered similar to the primes of the
natural numbers and are called “Gaussian primes.” Since we are considering
complex integers, we will be operating within the framework of the 2D plane
restricted to integers, so when we say “step right” that could be thought to
mean to move from the point $(x,y)$ to $(x+1,y)$ or $a+bi$ to $a+1+bi$.

\bigskip
We will investigate Gaussian prime cycles. The cycles are formed by starting
at a given point and stepping right until a Gaussian prime is found, at which
point the direction becomes down. The process of stepping in the same
direction until a Gaussian prime is found, then rotating the direction 90
degrees is repeated until we return to the initial point and initial
direction. For the project, we will plot the cycle length as a function of
the initial point and the discrete derivative/finite difference of the
function. The question we hope to answer is whether there is a discernable
pattern to the aforementioned plots. The project will approach the problem
using two implementations: numeric, using standard Python, and symbolic, using
the Python library sympy.

\end{document}
